\section{Conclusioni}
Lo scopo dell’esperienza era quello di verificare come la funzione di trasferimento fosse in grado di studiare il funzionamento di un circuito soltanto in base alla frequenza e di ricavare i valori delle diverse componenti dalla funzione stesse.
Per quanto concerne la parte dell'esperimento relativa ai circuiti RC ed RL, non siamo riusciti a risalire a una fonte di errore per l’incompatibilità tra il modello della fase e i dati raccolti; dubitiamo fortemente che l’errore possa essere legato alla 
procedura di misura, in quanto all’inizio dell’esperienza abbiamo compiuto una calibrazione delle sonde e durante gli esperimenti stessi ci siamo sempre accurati della correttezza del segnale ricevuto attraverso una valutazione qualitativa dell’andamento dell’ampiezza ricevuto dalla funzionalità MATH dell’oscilloscopio. Abbiamo, inoltre, tentato di ripetere i calcoli usando i massimi, anziché gli zeri, e abbiamo provato a non prendere la 
prima coppia di zeri disponibile, ma nessuno di questi metodi è riuscito a risolvere il nostro problema. Non siamo, quindi, riusciti a identificare la fonte di questo errore tra modello e dati 
presi e siamo quindi costretti a dichiarare non valida quella parte di esperimento. Al contrario, abbiamo ottenuto risultati soddisfacenti per il modello dato dal modulo della funzione 
di trasferimento, sia per il caso $V_{a} e V_{b}$ che per il caso $V_{a-b} e V_{b}$, con test del chi quadro 
accettabili per RC e RL.

Per RLC, invece, come accennato in precedenza, il $\chi ^{2}$ rivela incompatibilità tra modello e misure raccolte. Riteniamo, però, che questa inconciliabilità sia dovuto a un errore nell’esecuzione dell’algoritmo del fit dovuto alla posizione del massimo. 
Riteniamo che l’errore sia facilmente risolvibile ripetendo l’esperimento scegliendo componenti circuitali che ci permettano di osservare il massimo in una posizione più avanzata sulla scala delle $\Omega$, conoscendo infatti la relazione tra la posizione del massimo e LC, cioè: $\Omega_{max}$ = $\frac{1}{\sqrt{LC}}$, è facile riorganizzare il circuito al 
fine di spostare il massimo in una posizione più comoda.