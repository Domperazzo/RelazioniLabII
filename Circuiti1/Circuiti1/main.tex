\documentclass[a4paper,12pt, italian]{article}
\usepackage{geometry}
\geometry{a4paper, top=3cm, bottom=3cm, left=2.5cm, right=2.5cm, heightrounded, bindingoffset=5mm}
\usepackage[italian]{babel}
\usepackage[version=3]{mhchem} 
\usepackage{siunitx} 
\usepackage{graphicx}
\usepackage{booktabs}
\usepackage{natbib} 
\usepackage{amsmath}
\usepackage{hyperref}
%\usepackage[labelformat=empty]{caption}
\setlength\parindent{15pt}
\usepackage{natbib}
\usepackage{amscd}
\usepackage{siunitx}
\usepackage{booktabs}
\usepackage{multicol}
\usepackage{tikz}
\usepackage{amsmath,amssymb}
\usepackage{amsthm}
\usepackage{tikz}
\usepackage{cancel}
\usepackage{float}
\usepackage{placeins} %aggiungi questo package e usa \FloatBarrier all'inizio e alla fine della sequenza di tabelle
\usepackage[labelsep=space]{caption} 
\renewcommand{\labelenumi}{\alph{enumi}.} 
\newcommand{\minitab}[2][l]{\begin{tabular}#1 #2\end{tabular}}
\newcommand{\angstrom}{\mbox{\normalfont\AA}}

\title{\textsc{Verifica della Legge di Ohm\\
Studio delle resistenze\\
Caratterizzazione corrente-tensione di un dispositivo non lineare}}
\date{}

\author{Laura \textsc{Trombetta}\\ Alessandro Maria \textsc{Turturiello}\\Federico \textsc{Venturoli}} 



\begin{document}

\maketitle 

\begin{center}
\begin{tabular}{l r}
Eseguita il giorno: &  5 maggio 2022 \\ 
Gruppo T2A-4: & Trombetta Laura\\
& Turturiello Alessandro Maria \\
& Venturoli Federico \\ 

\end{tabular}
\end{center}
\begin{abstract}
Lo scopo dell'esperienza è stato quello di studiare il funzionamento di un circuito, attraverso la Legge di Ohm, a seconda della posizione di una o più resistenze. E' stato anche caratterizzato l'andamento tensione-corrente di un dispositivo non lineare, nel nostro caso un diodo.
\end{abstract}
\tableofcontents
\maketitle
\newpage

\section{Introduzione e descrizione dell'apparato sperimentale}
L'apparato sperimentale è costituito da un emettitore e un ricevitore di microonde, montati su un metro lungo il quale possono scorrere. Al centro del metro è disposto un goniometro che permette ai due bracci del metro di ruotare e di ricavarne il relativo angolo. Il goniometro permette l'inserimento di molteplici componenti (pedane e piedistalli di supporto) attraverso un gancio. L'emettitore e il ricevitore possono ruotare di angoli osservabili attraverso altri due goniometri e l'intensità delle onde trasmesse è rilevata da un amperometro, con scala regolabile.
Le componenti aggiuntive consistono in una pedana di supporto, necessaria per condurre esperienze come quella di Bragg o di Brewster, e componenti mobili muniti di magneti necessari per fissare lenti e specchi riflettenti.

Prima di procedere con l'esperimento abbiamo condotto un'esperienza virtuale dalla quale abbiamo tratto delle considerazioni.

La principale osservazione è stata in relazione alla polarizzazione in quanto, in modo analogo a quanto mostrato nella simulazione, il segnale rilevato sull'amperometro è provocato da due grandezze, la cui relazione reciproca è ignota. Nell'esperimento reale ciò è causato dal fatto che i coni di ricevitore ed emettitore non sono simmetrici per rotazione.
\section{Misura della caratteristica corrente-tensione di un resistore}
\subsection{Misura resistenza Voltmetro e Amperometro}
Per verificare che i lettori di grandezza di cui disponevamo fossero ben progettati, abbiamo proceduto analizzando due configurazoni differenti di circuiti.

Nella prima configurazione abbiamo disposto gli elementi circuitali come in figura:

\begin{figure}[h!]
    \centering
    \includegraphics[scale=1]{Immagini/Conf1.PNG}
    \label{fig:my_label}
\end{figure}

Un lettore di tensione ideale dispone di una resistenza infinita; poichè nel caso reale non è possibile riporodurre questo fenomeno, si scelgono per i circuiti resistenze di ordini di grandezza molto minori di quelle dello strumento.
Abbiamo scelto la resistenza R, disposta in parallelo con quella del Voltmetro $R_{V}$, con un valore di $R=22M\Omega$. Questa scelta è stata motivata dal fatto che, per poter in seguito utilizzare resistenze con valori adatti, è stato necessario misurare quella del Voltmetro, confrontandola con un resistore che avesse un ordine di grandezza simile.
Riportiamo di seguito i dati raccolti, utilizzati per condurre un'interpolazione.

\begin{table}[H]
    \centering
    \begin{tabular}{cc}
    \toprule
    \Delta V (V)  & I (\mu A) \\
    \midrule
    0,507	&0,59\\
    1,012	&1,20\\
    1,527	&1,83\\
    2,032	&2,43\\
    2,544	&3,04\\
    3,051	&3,65\\
    3,56	&4,27\\
    \bottomrule
    \end{tabular}
    \label{tab:my_label}
\end{table}
 
\begin{figure}[H]
    \centering
    %\includegraphics[scale=.5]{Immagini/Ohm configurazione 1.pdf}
    \includegraphics[scale=.5]{Immagini/fit1.pdf}
    \label{fig:my_label}
    \caption{Grafico configurazione Voltmetro}
\end{figure}

La resistenza del Voltmetro risulta essere di $1,05M\Omega$, essa dovrebbe, però, essere dell'ordine di grandezza di 10 $M\Omega$; discuteremo i motivi di questa incongruenza nelle conclusioni, ma crediamo possa essere attribuibile alla resistenza nota scelta.

Per il calcolo dell'errore abbiamo operato come segue:

$$
\sigma_{R_v}^2=\left[\sigma_{R_{equiv}}\left(\dfrac{\partial R_v}{\partial R_{equiv}}\right)\right]^2=\sigma^2_{R_{equiv}}\left(\dfrac{R_v^2}{(R_n -R_{tot}^2)}\right)^2=2,322 \Omega
\qquad \text{}
$$


Nella seconda configurazione abbiamo disposto gli elementi circuitali come in figura:

\begin{figure}[H]
    \centering
    \includegraphics[scale=1]{Immagini/Conf2.PNG}
    \label{fig:my_label}
\end{figure}

Abbiamo scelto la resistenza R, disposta in serie con quella del Amperometro $R_{A}$, con un valore di $R=10\Omega$ per un motivo simile al precedente. A differenza del Voltmetro, l'Amperometro ideale ha una resistenza nulla, quindi, nella realtà, si scelgono resistori di ordini di grandezza molto maggiori dello strumento.
Anche in questo caso abbiamo deciso di condurre un'interpolazione per ricavare la resistenza equivalanente e, successivamente, quella dello strumento.

\begin{table}[H]
    \centering
    \begin{tabular}{cc}
    \toprule
    \Delta V (V)  & I (mA) \\
    \midrule
    1,016	&88,56\\
    1,515	&132,06\\
    2,026	&176,42\\
    2,533	&220,37\\
    3,044	&264,4\\
    3,549	&307,59\\
    4,061	&351,3\\
    \bottomrule
    \end{tabular}
    \label{tab:my_label}
\end{table}

\begin{figure}
    \centering
    %\includegraphics[scale=.5]{Immagini/Ohm configurazione 2.pdf}
    \includegraphics[scale=.5]{Immagini/fit2.pdf}
    \label{fig:my_label}
    \caption{Grafico configurazione Amperometro}
\end{figure}

Questo risultato risulta congruente con quello che ci aspettavamo.

\subsection{Verifica della legge di Ohm}
Successivamente, configurando correttamente entrambi i circuiti, abbiamo verificato la Legge di Ohm. Variando la misura della tensione di alimentazione del circuito, abbiamo misurato la differenza di potenziale ai capi del resistore e la corrente che lo attraversava.

\begin{table}[H]
\parbox{.45\linewidth}{
    \centering
    \begin{tabular}{cc|cc}
    \toprule
    \Delta V (V)  & I (mA) & \Delta V (V)  & I (mA) \\
    \midrule
    0,456	&0,45693 & 1,385	&1,3869\\
    0,55	&0,5497 & 1,474	&1,4761\\
    0,649	&0,6498 & 1,572	&1,5741\\
    0,736	&0,7372& 1,658	&1,6604\\
    0,828	&0,8293& 1,755	&1,7584\\
    0,927	&0,928& 1,847	&1,85\\
    1,015	&1,0166& 1,937	&1,9403\\
    1,103	&1,1048& 2,03	&2,0335\\
    1,203	&1,2048& 2,124	&2,128\\
    1,291	&1,293& 2,216	&2,2196\\
    2,308	&2,3126&&\\
    \bottomrule
    \end{tabular}
    \caption{Dati configurazione 1, R=1k\Omega}
    \label{tab:my_label}
}
\quad
\parbox{.45\linewidth}{
    \centering
    \begin{tabular}{cc|cc}
    \toprule
    \Delta V (V)  & I (\mu A) & \Delta V (V)  & I (\mu A) \\
    \midrule
    1,017	&1,12& 2,029	&2,24\\
    1,117	&1,22&2,134	&2,36\\
    1,219	&1,35& 2,237	&2,47\\
    1,319	&1,45&2,338	&2,59\\
    1,423	&1,57&2,436	&2,7\\
    1,524	&1,68&2,538	&2,81\\
    1,627	&1,8&2,646	&2,93\\
    1,727	&1,9&2,738	&3,03\\
    1,831	&2,02&2,85	&3,16\\
    1,927	&2,14&2,948	&3,27\\
    3,053	&3,37 &&\\
    \bottomrule
    \end{tabular}
    \caption{Dati configurazione 2, R=0,91M\Omega}
    \label{tab:my_label}
    }
\end{table}

\begin{figure}[H]
    \centering
    \includegraphics[scale=.45]{Immagini/fit3.pdf}
    \quad
    \includegraphics[scale=.45]{Immagini/fit4.pdf}
    \caption{Configurazione 1 in alto e configurazione 2 in basso}
\end{figure}


\subsection{Misura di resistenze composite}
Infine abbiamo disposto, prima in parallelo e poi in serie, due resistori (scelti con lo stesso ordine di grandezza della resistenza) e abbiamo verificato con essi la legge di Ohm. In questo caso abbiamo utilizzato solamente la configurazione 2, in quanto la misura che in precedenza avevamo ottenuto per $R_{V}$ poteva non essere corretta.

\begin{table}[H]
\parbox{.45\linewidth}{
    \centering
    \begin{tabular}{cc}
    \toprule
    $\Delta$ V (V)  & I (mA) \\
    \midrule
    0,503	&44,12\\
1,016	&89,15\\
1,519	&133,25\\
2,029	&177,94\\
2,538	&221,84\\
3,044	&265,8\\
3,551	&109,5\\
4,055	&352,8\\
    \bottomrule
    \end{tabular}
    \caption{Resistenze in parallelo; R_{tot}=5$\Omega$}
    \label{tab:my_label}
    }
    \quad
    \parbox{.45\linewidth}{
    \centering
    \begin{tabular}{cc}
    \toprule
    $\Delta$ V (V)  & I (mA) \\
    \midrule
    0,507	&23,916\\
1,016	&47,931\\
1,524	&71,92\\
2,029	&95,73\\
2,539	&119,76\\
3,048	&114,7\\
3,551	&167,36\\
4,065	&191,45\\
    \bottomrule
    \end{tabular}
    \caption{Resistenze in serie; R_{tot}=20$\Omega$}
    \label{tab:my_label}
    }
\end{table}
\noindent
I fit e i relativi risultati sono riportati nella Figura \ref{fit 56}.
\begin{figure}[h!]
    \centering
    \includegraphics[scale=.4]{Immagini/fit5.pdf}
    \\
    \includegraphics[scale=.4]{Immagini/fit6.pdf}
    \caption{}
    \label{fit 56}
\end{figure}
\section{Partitore resistivo}
In questa parte di esperienza abbiamo studiato la seguente configurazione:

\begin{figure}[H]
    \centering
    \includegraphics[scale=1]{Immagini/PartResist.PNG}
    \label{fig:my_label}
\end{figure}

nella quale con $R_{load}$ indichiamo il partitore resistivo, componente di circuito di cui è possibile scegliere la resistenza attraverso interruttori.
Abbiamo studiato la configurazione sotto la condizione in cui $V_{out}$ = 0.5$V_{in}$.
Attraverso considerazioni sulle resistenze equivalenti e sulla confrontabilità tra la resistenza del partitore e quelle note, siamo arrivati alla conclusione per la quale, al fine di avere una configurazione con la condizione sopra citata, era necessario che $R_{1}$ ed $R_{2}$ fossero uguali e che $R_{load}$ fosse mandata a infinito. Nella pratica, questa procedura è stata compiuta aumentando sempre più la resistenza del partitore.

\begin{equation}
R_{eq} = \frac{R_{2}R_{load}}{R_{2}+R_{load}} \hspace{73 pt} R_{tot} = R_{1}+\frac{R_{2}R_{load}}{R_{2}+R_{load}}
\end{equation}
\begin{equation}
\label{eq: 5}
V_{in} = R_{tot}I \hspace{73 pt} V_{out} = R_{eq}I = V_{in}\frac{R_{eq}}{R_{1}+R_{eq}}  
\end{equation}
\noindent
Dalla equazione \ref{eq: 5} è evidente che, nel caso in cui  $R_{load}\to \infty$ e $V_{out}$ = 0.5$V_{in}$, si ottiene \begin{equation}
\frac{1}{2} = \frac{R_{2}}{R_{1}+R_{2}} \hspace{20 pt} \text{per cui} \hspace{20 pt} R_{1} = R_{2}
\end{equation}

Abbiamo confermato queste osservazioni sperimentalmente (utilizzando come resistenze $R_{1} = R_{2} = 150k\Omega)$, come risulta visibile dal grafico:

\begin{figure}[H]
    \centering
    \includegraphics[scale=.6]{partitore resistivo.pdf}
    \caption{}
    \label{fig: partitore resistivo}
\end{figure}
\section{Misura della caratteristica corrente-tensione di un diodo}
In questa sezione abbiamo sostituito la resistenza nel circuito utilizzato in precedenza con un diodo. Esso è un elemento circuitale non lineare la cui funzione è quella di permettere alla corrente elettrica di fluire in un verso e di bloccarla quasi totalmente nell'altro.
Abbiamo proceduto variando la tensione di alimentazione del circuito, misurando la differenza di potenziale ai capi del diodo e la corrente che lo attraversava.
Nonostante la Legge di Shockley descriva l'andamento della corrente con una relazione esponenziale, nella realtà si usa definire una tensione di soglia, $V_{soglia}$, oltre la quale il diodo è considerato in conduzione.
In primis abbiamo verificato la legge sopra citata attraverso un fit esponenziale, in cui l'unico parametro era $I_{0}$, in quanto gli altri erano costanti o termini noti.
Di seguito è riportato il fit esponenziale

\begin{figure}[H]
    \centering
    \includegraphics[scale=.5]{Immagini/diodo2.pdf}
    \caption{Fit con la legge di Shockley}
\end{figure}
In secondo luogo abbiamo individuato la $V_{soglia}$, fittando i dati con una retta, partendo dal range superiore delletensioni, fino a raggiungere un $\chi ^{2}$ ridotto di circa 1. Trovando l'intercetta di questaretta con l'asse \textit{x}, abbiamo ricavato $V_{soglia}$. Inoltre il parametro $p_0$ indica la corrente di saturazione che risulta essere dell'ordine del $10^{-13}$ come previsto dal modello teorico, mentre il parametro $p_1$ esprime il valore del rapporto $\frac{1}{g}$ dove $g$ è una costante che dipende dal materiale componente il diodo, pari circa a $1$ per diodi in germanio mentre circa $2$ per diodi in silicio. Da questo dato ricaviamo che $g\approx 1$ per cui l'ipotesi che il diodo sia composto da germanio risulta essere giustificata.
Di seguito è riportato il grafico con il fit della retta:

\begin{figure}[h!]
    \centering
    \includegraphics[scale=.5]{Immagini/diodo_soglia2.pdf}
    \caption{Fit con il diodo in conduzione da cui si ricava $V_{\text{soglia}}$ che corrisponde con il parametro $P_0$}
    \label{fig:my_label}
\end{figure}
\section{Conclusioni}
Lo scopo di questa esperienza è stato quello di studiare le relazioni che governo circuiti composti da condensatori, induttanze e resistenze. Abbiamo studiato circuiti RC, RL e RLC in tutti i suoi stati possibili. Il fine delle misurazioni era quello di riottenere i valori di C e L attraverso la verifica delle leggi ottenute dallo studio teorico sui circuiti. L'unica configurazione in cui abbiamo riscontrato difficoltà a fittare i dati è stata quella del circuito RC, in cui abbiamo, quindi, optato per un fit in scala logaritmica. Il circuito RC è anche l'unico in cui il valore stimato di C, non corrisponde con il valore che potevamo aspettarci, in quanto doveva essere dell'ordine dei nanofarad. Riteniamo che la causa di questo errore possa essere attribuita sia alla scelta della resistenza sia alla maniera con cui abbiamo interpolato i dati, in quanto abbiamo dovuto selezionare una parte delle misure collezionate. Infine abbiamo condotto un \textit{t}-test per verificare la confrontabilità dei valori di L e C ricavati dalle diverse configurazione RC/RL e RLC nei sui diversi stati.
Abbiamo condotto dei t-test per verificare la confrontabilitá dei diversi metodi per l'acquisizione dei parametri. Nel caso di L, nonostante solo un t-test risulti accettabile con una distanza di 4 sigma tra i 2 valori, nelle configurazioni risultano grandezze dell'ordine di grandezza aspettato (una decina di mH). La legge risulta sempre verificata, tranne nel caso nello sovrasmorzamento, nel quale riteniamo che l'errore nel fit sia dovuto ad un errore dell'algoritmo, a causa della quantità di dati.
Diversamente avviene per C, in cui solo in 2 casi i risultati sono nell'ordine i grandezza aspettato (circa tra $10^{-7}$ e $10^{-9}$). La causa di tale errore può essere riconducibile a due diversi motivi. In primis, crediamo che in configurazioni come RC e RLC (sovrasmorzamento) l'utilizzo di una diversa resistenza possa migliorare l'accuratezza e la precisione delle misure. Riteniamo, in secondo luogo, anche che la scelta della scala dei tempi utilizzata per prendere le misurazioni tramite la chiavetta possa essere modificata per ottenere misure più precise.


\end{document}