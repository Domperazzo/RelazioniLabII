\section{Conclusioni}
Lo scopo di questa esperienza è stato quello di studiare e verficare relazioni e leggi riguardanti resistenze e diodi. In primo luogo è stata condotta una verifica sulle due configurazioni, quella con il voltmetro e quella con l'amperometro, al fine di analizzare i loro comportamenti rispetto alle configurazioni ideali. Per l'amperometro i risultati ottenuti risultano congruenti con quelli aspettati, cosa che invece non accade con il voltmetro. Crediamo che la causa di ciò possa essere dovuta a due principali fattori. In primis riteniamo possibile che la resistenza scelta, la quale era necessario fosse confrontabile con quella del voltmetro, potrebbe non essere stata tale e che, quindi, l'errore sull'ordine di grandezza potrebbe essere stato causato da quello. Come seconda fonte di errore riconduciamo il fatto che, per identificare le resistenze, ci siamo serviti del tool online menzionato nell'introduzione, i cui valori potrebbero non corrispondere a quelli reali, inserendo quindi un possibile errore sistematico nei nostri risultati. A causa di tale incongruenza, in esperienze che fornivano la possibilità di scegliere la configurazione, abbiamo sempre preferito utilizzare quella con l'amperometro. Lo studio sulla relazione evidenziata da Ohm risulta corretta, nonostante il $\chi^2$ risulti molto bassa. La causa più probabile di questo fenomeno potrebbe essere ricercata in una sottostima degli errori, dovuta al nostro modo di identificare i valori delle resistenze. Nonostante questa possibile fonte di errore, lo studio sul partitore resistivo ha portato risultati congruenti rispetto al nostro studio teorico riguardo alle relazioni sulle resistenze. Infine, anche l'analisi sul diodo risulta congruente rispetto alla relazione ipotizzata, con un $\chi^2$ ridotto di circa 0.82. Riteniamo che l'esperimento potrebbe essere riprodotto con risultati più soddisfacenti se si procedesse a misurare le resistenze in maniera differente. 