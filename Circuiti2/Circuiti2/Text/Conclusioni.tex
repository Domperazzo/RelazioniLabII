\section{Conclusioni}
Lo scopo di questa esperienza è stato quello di studiare le relazioni che governo circuiti composti da condensatori, induttanze e resistenze. Abbiamo studiato circuiti RC, RL e RLC in tutti i suoi stati possibili. Il fine delle misurazioni era quello di riottenere i valori di C e L attraverso la verifica delle leggi ottenute dallo studio teorico sui circuiti. L'unica configurazione in cui abbiamo riscontrato difficoltà a fittare i dati è stata quella del circuito RC, in cui abbiamo, quindi, optato per un fit in scala logaritmica. Il circuito RC è anche l'unico in cui il valore stimato di C, non corrisponde con il valore che potevamo aspettarci, in quanto doveva essere dell'ordine dei nanofarad. Riteniamo che la causa di questo errore possa essere attribuita sia alla scelta della resistenza sia alla maniera con cui abbiamo interpolato i dati, in quanto abbiamo dovuto selezionare una parte delle misure collezionate. Infine abbiamo condotto un \textit{t}-test per verificare la confrontabilità dei valori di L e C ricavati dalle diverse configurazione RC/RL e RLC nei sui diversi stati.
Abbiamo condotto dei t-test per verificare la confrontabilitá dei diversi metodi per l'acquisizione dei parametri. Nel caso di L, nonostante solo un t-test risulti accettabile con una distanza di 4 sigma tra i 2 valori, nelle configurazioni risultano grandezze dell'ordine di grandezza aspettato (una decina di mH). La legge risulta sempre verificata, tranne nel caso nello sovrasmorzamento, nel quale riteniamo che l'errore nel fit sia dovuto ad un errore dell'algoritmo, a causa della quantità di dati.
Diversamente avviene per C, in cui solo in 2 casi i risultati sono nell'ordine i grandezza aspettato (circa tra $10^{-7}$ e $10^{-9}$). La causa di tale errore può essere riconducibile a due diversi motivi. In primis, crediamo che in configurazioni come RC e RLC (sovrasmorzamento) l'utilizzo di una diversa resistenza possa migliorare l'accuratezza e la precisione delle misure. Riteniamo, in secondo luogo, anche che la scelta della scala dei tempi utilizzata per prendere le misurazioni tramite la chiavetta possa essere modificata per ottenere misure più precise.