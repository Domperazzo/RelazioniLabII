\documentclass[a4paper,12pt, italian]{article}
\usepackage{geometry}
\geometry{a4paper, top=3cm, bottom=3cm, left=3cm, right=3cm, heightrounded, bindingoffset=5mm}
\usepackage[italian]{babel}
\usepackage[version=3]{mhchem} 
\usepackage{siunitx} 
\usepackage{graphicx}
\usepackage{booktabs}
\usepackage{natbib} 
\usepackage{amsmath}
\usepackage{hyperref}
%\usepackage[labelformat=empty]{caption}
\setlength\parindent{15pt}
\usepackage{natbib}
\usepackage{amscd}
\usepackage{siunitx}
\usepackage{booktabs}
\usepackage{multicol}
\usepackage{tikz}
\usepackage{amsmath,amssymb}
\usepackage{amsthm}
\usepackage{tikz}
\usepackage{cancel}
\usepackage{placeins} %aggiungi questo package e usa \FloatBarrier all'inizio e alla fine della sequenza di tabelle
\usepackage[labelsep=space]{caption} 
\renewcommand{\labelenumi}{\alph{enumi}.} 
\newcommand{\minitab}[2][l]{\begin{tabular}#1 #2\end{tabular}}

\title{\textsc{Studio delle leggi che governano le inteterferenze con sorgenti nello spettro delle microonde}}
\date{}

\author{Laura \textsc{Trombetta}\\ Alessandro Maria \textsc{Turturiello}\\Federico \textsc{Venturoli}} 



\begin{document}

\maketitle 

\begin{center}
\begin{tabular}{l r}
Eseguita il giorno: &  31 marzo 2022 \\ 
Gruppo T2A-4: & Trombetta Laura\\
& Turturiello Alessandro Maria \\
& Venturoli Federico \\ 

\end{tabular}
\end{center}
\begin{abstract}
    Lo scopo della relazione è lo studio delle leggi dell'ottica, grazie all'utilizzo di un'onda elettromagnetica che rientra nel range delle microonde.
    Abbiamo condotto molteplici esperimenti per verificare le leggi di polarizzazione, riflessione, rifrazione, interferenza e la natura stazionaria dell'onda in specifiche condizioni. Gli esperimenti hanno avuto nel complesso risultati soddisfacenti, in caso contrario siamo riusciti ad identificare la sorgente dell'errore, in modo da non invalidare la legge che stavamo verificando. 
\end{abstract}
\tableofcontents
\maketitle
\newpage

\section{Introduzione e descrizione dell'apparato sperimentale}
L'apparato sperimentale è costituito da un emettitore e un ricevitore di microonde, montati su un metro lungo il quale possono scorrere. Al centro del metro è disposto un goniometro che permette ai due bracci del metro di ruotare e di ricavarne il relativo angolo. Il goniometro permette l'inserimento di molteplici componenti (pedane e piedistalli di supporto) attraverso un gancio. L'emettitore e il ricevitore possono ruotare di angoli osservabili attraverso altri due goniometri e l'intensità delle onde trasmesse è rilevata da un amperometro, con scala regolabile.
Le componenti aggiuntive consistono in una pedana di supporto, necessaria per condurre esperienze come quella di Bragg o di Brewster, e componenti mobili muniti di magneti necessari per fissare lenti e specchi riflettenti.

Prima di procedere con l'esperimento abbiamo condotto un'esperienza virtuale dalla quale abbiamo tratto delle considerazioni.

La principale osservazione è stata in relazione alla polarizzazione in quanto, in modo analogo a quanto mostrato nella simulazione, il segnale rilevato sull'amperometro è provocato da due grandezze, la cui relazione reciproca è ignota. Nell'esperimento reale ciò è causato dal fatto che i coni di ricevitore ed emettitore non sono simmetrici per rotazione.
\section{Onde stazionarie}
Obiettivo di questa sezione è stato quello di misurare la lunghezza d'onda $\lambda $ di una microonda, il cui range di grandezza varia tra 1 mm e 30 cm.

Abbiamo preferito procedere contando il numero di massimi, corrispondenti al punto di inversione della lancetta dell'amperometro, per una lunghezza di d = 20 cm. Questa scelta è stata motivata dal fatto che preferire, invece, un passo piccolo avrebbe aumentato l'imprecisione. Quest'ultima è dovuta alla presenza di componenti sinusoidali aggiuntive che sporcano l'onda non perfettamente stazionaria.

Per ricavare il valore della lunghezza d'onda ci siamo serviti della relazione 
\begin{equation}
    \lambda = \dfrac{2\, d}{\text{Numero di massimi}}
\end{equation}
di seguito abbiamo riportato le lunghezze d'onda calcolate e il relativo errore ricavato dalla propagazione degli errori.

\begin{table}[h!]
    \centering
    \begin{tabular}{ccc}
        Numero massimi & $\lambda$ (cm) & errore $\lambda$\\
        \hline
    14&	2,857&	0,014\\
    13&	3,077&	0,015\\
    14&	2,857&	0,014\\
    14&	2,857&	0,014\\
    13&	3,077&	0,015\\
    13&	3,077&	0,015\\
    14&	2,857&	0,014\\
    14&	2,857&	0,014\\
    14&	2,857&	0,014\\
    14&	2,857&	0,014\\
    14&	2,857&	0,014\\
    14&	2,857&	0,014\\
    14&	2,857&	0,014\\
    14&	2,857&	0,014\\
    14&	2,857&	0,014\\
    14&	2,857&	0,014\\
\hline\hline
    \end{tabular}
    \caption{}
    \label{lunghezze donda}
\end{table}
\noindent
Il valore ottenuto è di 
$$
\lambda = (2,898 \pm 0,014)\,cm
$$
il cui errore è ricavato dalla media dei singoli errori.

\section{Riflessione e rifrazione}
\subsection{Riflessione}
Per analizzare il fenomeno della riflessione abbiamo verificato la legge di Cartesio (riportata sotto), misurando gli angoli di incidenza e riflessione della microonda su uno specchio riflettente.
\begin{equation}
    \theta_i=\theta_r
\end{equation}
\noindent
Gli angoli ricavati, misurati in gradi, sono i seguenti:

\begin{table}[h!]
    \centering
    \begin{tabular}{cccc}
    $\theta_i \, (^\circ)$ && \,$\theta_r \, (^\circ)$&\\
    \toprule
    20	&20	&&21\\
    20	&21	&&23\\
    30	&31	&&31\\
    30	&29	&&27\\
    40	&37	&&35\\
    40	&35	&&39\\
    50	&43	&&48\\
    50	&44	&&43\\
    60	&65	&&69\\
    60	&63	&&63\\
    \bottomrule
    \end{tabular}
    \caption{}
    \label{}
\end{table}

Abbiamo osservato che per angoli piccoli l'errore era maggiore, supponendo che ciò fosse dovuto al fatto che emettitore e ricevitore si trovassero troppo allineati. Probabilmente il ricevitore captava il segnale non solo dell'onda riflessa ma anche di quella emessa.
\subsection{Rifrazione}
Abbiamo studiato il fenomeno della rifrazione con lo scopo di misurare l'indice di rifrazione dello styrene, un idrocarburo aromatico.
Abbiamo posizionato un contenitore di polistirolo a base triangolare sulla pedana al centro. Dopo aver verificato che l'indice di rifrazione del polistirolo fosse pari a quello dell'aria abbiamo misurato l'angolo tra l'ipotenusa e il cateto maggiore attraverso le formule trigonometriche, l'angolo risultava essere di $\theta_v = 22,54 ^\circ$.

\begin{figure}[h!]
    \centering
    \includegraphics[scale=.7]{Immagini/triangolo.png}
    \caption{}
    \label{triangolo}
\end{figure}
 
In seguito abbiamo preso l'angolo formato con la normale, $\theta_n = 38,54 ^\circ$, da cui abbiamo ricavato il valore di n = 1,63 come indice di rifrazione dello styrene. Quest'ultimo è stato calcolato attraverso la seguente relazione:
$$
n=\dfrac{\sin\theta_n}{\sin\theta_v}
$$

\section{Polarizzazione}
Questa sezione ha come obiettivo quello di determinare la relazione tra intensità e angolo di polarizzazione.
Per condurre questa esperienza abbiamo disposto emettitore e ricevitore allineati e abbiamo ruotato il ricevitore di diversi angoli $\theta$.
Tenendo conto dell'imprecisione messa in evidenza nell'introduzione, abbiamo deciso di rappresentare la relazione in due modi differenti.
In primo luogo abbiamo utilizzato un'interpolazione lineare sul $\cos^2\vartheta$, verificando che l'apparato non avesse errore di calibrazione in base al verso della rotazione. I risultati sono riportati nella Figura \ref{polatizzazione fit quadratico}.
\begin{figure}[h!]
    \centering
    \includegraphics[scale=.5]{Immagini/coseni quadri lineare.pdf} 
    \caption{Interpolazione lineare}
    \label{polatizzazione fit quadratico}
\end{figure}
Abbiamo dimostrato questo mostrando come l'intensità fosse la medesima per angoli positivi e negativi di uguale modulo.

In secondo luogo abbiamo fatto uso di un'interpolazione quadratica che segue la legge di Malus, in grado di descrivere l'imprecisione sulla relazione tra il segnale rilevato e l'effettivo segnale trasmesso. Abbiamo interpolato seguendo 
$$
y=A\cos(\theta) + B(\cos^2\theta)
$$
dove per $y$ si considera il segnale misurato, il grafico è riportato nella Figura \ref{polatizzazione fit}.

\begin{figure}[h!]
    \centering
    \includegraphics[scale=.5]{Immagini/coseni quadri.pdf}
    \caption{Interpolazione completa}
    \label{polatizzazione fit}
\end{figure}
Non essendo nota la relazione tra il segnale letto sul display dell'Amperometro e il valore medio del campo elettrico nel punto del ricevitore, abbiamo ipotizzato che questo avesse, nel primo caso, una dipendenza lineare dal campo, nel secondo caso, invece, una dipendenza "mista" sia dal campo che dall'intensità.

\section{Angolo di Brewster}
\begin{figure}[h!]
    \centering
    \includegraphics[scale=.3]{Immagini/Schermata 2022-04-07 alle 2.36.29 PM.png}
    \caption{Configurazione dell'apparato sperimentale per la misura dell'angolo di Brewster}
    \label{apparato browser}
\end{figure}

Obiettivo della sezione è stato quello di verificare l'angolo di Brewster.
\noindent
Abbiamo campionato per diversi angoli il valore dell'intensità, con maggiore concentrazione di campionamenti vicino al massimo.
\begin{figure}[h!]
    \centering
    \includegraphics[scale=.5]{Immagini/browser.pdf}
    \caption{}
    \label{}
\end{figure}
Dopo aver interpolato i dati vicino a quest'ultimo, come se questi seguissero una relazione quadratica, abbiamo ricavato il massimo attraverso il calcolo analitico della derivata. Il valore ricavato è $\ang{64}\, 86'$.
Per valutare l'errore abbiamo tenuto conto della correlazione tra $A$ e $B$, calcolando la covarianza tra i due parametri. Covarianza ricavata dalla matrice di covarianza (Eq. \ref{eq 3}) dei tre parametri utilizzati per l'interpolazione.
\begin{equation}
\sigma_{ABC}=
\begin{pmatrix}
5.5232\cdot 10^{-10} & -6.5443\cdot 10^{8} &
1.7909\cdot 10^{-6}\\
\\
-6.5443\cdot 10^{-8} &
7.9896\cdot 10^{-6} & -2.2544\cdot 10^{-4}\\
\\
1.7909\cdot 10^{-6} & -2.2544\cdot 10^{-4} &
6.6125\cdot 10^{-3}\\
\end{pmatrix}
\label{eq 3}
\end{equation}

%La covarianza risulta essere di METTERE VALORE COVARIANZA, l'errore invece di METTERE VALORE ERRORE.
%fare propagazione errore sull'angolo calcolato a mano

\section{Interferenza}
Per analizzare il fenomeno dell'interferenza abbiamo predisposto l'apparato differentemente per due diverse esperienze.

\subsection{Esperienza di Michelson}
In primis abbiamo disposto le lenti e gli specchi come riportato in Figura \ref{michelino}

\begin{figure}[h!]
    \centering
    \includegraphics[scale=.35]{Immagini/michelino.png}
    \caption{Configurazione di Michelson}
    \label{michelino}
\end{figure}
\noindent
Mantenendo tutte le componenti fisse, abbiamo proceduto spostando una delle lastre semi-riflettenti con intervalli di $\Delta d = 10 cm$. Dopo aver contato il numero di massimi per tale spostamento, abbiamo calcolato la relativa lunghezza d'onda $\lambda$, utilizzando la seguente relazione:

\begin{equation}
    \lambda=2\dfrac{L_{M1} + \Delta d - L_f}{\Delta N}
\end{equation}
dove $L_{M1}$ indica la posizione iniziale dello specchio mobile e $L_F$ indica la posizione iniziale dello specchio fisso
\noindent
Riportiamo di seguito i valori ricavati:

\begin{table}[h!]
    \centering
    \begin{tabular}{c|cc}
    Numero di massimi & $\lambda$ $(cm)$ & $\sigma_\lambda$ ($cm$)\\
    \hline
6	&	3,33	&0,06\\
6	&	3,33	&0,06\\
7	&	2,86	&0,05\\
7	&	2,86	&0,05\\
7	&	2,86	&0,05\\
7	&	2,86	&0,05\\
7	&	2,86	&0,05\\
7	&	2,86	&0,05\\
8	&	2,50	&0,04\\
7&		2,86&	0,05\\
6	&	3,33	&0,06\\
6&		3,33&	0,06\\
7&		2,86&	0,05\\
7	&	2,86	&0,05\\
7	&	2,86	&0,05\\
7	&	2,86	&0,05\\
7	&	2,86	&0,05\\
\hline\hline
    \end{tabular}
    \caption{Caption}
    \label{tab:my_label}
\end{table}
\noindent
Il valore medio della lunghezza d'onda e il relativo errore, calcolato come media di ogni errore calcolato con il metodo della propagazione degli errori, è di 
$$
\lambda = 2,948 \pm 0,051\, cm
$$
\subsection{Esperienza di Lloyd}
Anche il questo caso lo scopo dell'esperimento è stato quello di determinare la lunghezza d'onda attraverso lo studio del fenomeno dell'interferenza.

Predisposte le componenti come in Figura \ref{loyd} abbiamo cercato di rilevare la posizione dei massimi a partire dal primo. Tale operazione è risultata difficile a causa delle numerose fonti d'incertezze. 

\begin{figure}[h!]
    \centering
    \includegraphics[scale=.35]{Immagini/loid.png}
    \caption{}
    \label{loyd}
\end{figure}
\noindent
I dati raccolti sono i seguenti:
\begin{table}[h!]
    \centering
    \begin{tabular}{ccc}
        Numero del massimo & $d$ $(cm)$ & $\lambda$ $(cm)$ \\
        \hline 
    1&	3,4&1,0	\\
    2&	4,4&5,1	\\
    3&	9,5&2,5	\\
    4&	12&	   4,0 \\
    5&	16&	    8,1\\
    6&	24,1&	\\
\hline\hline
\end{tabular}
\end{table}

E' ben visibile l'errore che traspare dalle misurazioni effettuate, esso verrà discusso nel dettaglio all'interno delle conclusioni.

\section{Conclusioni}
Lo scopo di questa esperienza è stato quello di studiare le relazioni che governo circuiti composti da condensatori, induttanze e resistenze. Abbiamo studiato circuiti RC, RL e RLC in tutti i suoi stati possibili. Il fine delle misurazioni era quello di riottenere i valori di C e L attraverso la verifica delle leggi ottenute dallo studio teorico sui circuiti. L'unica configurazione in cui abbiamo riscontrato difficoltà a fittare i dati è stata quella del circuito RC, in cui abbiamo, quindi, optato per un fit in scala logaritmica. Il circuito RC è anche l'unico in cui il valore stimato di C, non corrisponde con il valore che potevamo aspettarci, in quanto doveva essere dell'ordine dei nanofarad. Riteniamo che la causa di questo errore possa essere attribuita sia alla scelta della resistenza sia alla maniera con cui abbiamo interpolato i dati, in quanto abbiamo dovuto selezionare una parte delle misure collezionate. Infine abbiamo condotto un \textit{t}-test per verificare la confrontabilità dei valori di L e C ricavati dalle diverse configurazione RC/RL e RLC nei sui diversi stati.
Abbiamo condotto dei t-test per verificare la confrontabilitá dei diversi metodi per l'acquisizione dei parametri. Nel caso di L, nonostante solo un t-test risulti accettabile con una distanza di 4 sigma tra i 2 valori, nelle configurazioni risultano grandezze dell'ordine di grandezza aspettato (una decina di mH). La legge risulta sempre verificata, tranne nel caso nello sovrasmorzamento, nel quale riteniamo che l'errore nel fit sia dovuto ad un errore dell'algoritmo, a causa della quantità di dati.
Diversamente avviene per C, in cui solo in 2 casi i risultati sono nell'ordine i grandezza aspettato (circa tra $10^{-7}$ e $10^{-9}$). La causa di tale errore può essere riconducibile a due diversi motivi. In primis, crediamo che in configurazioni come RC e RLC (sovrasmorzamento) l'utilizzo di una diversa resistenza possa migliorare l'accuratezza e la precisione delle misure. Riteniamo, in secondo luogo, anche che la scelta della scala dei tempi utilizzata per prendere le misurazioni tramite la chiavetta possa essere modificata per ottenere misure più precise.

\end{document}