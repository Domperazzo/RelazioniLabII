\section{Conclusioni}
Lo scopo dell'esperimento era quello di studiare una microonda attraverso l'osservazione di diversi fenomeni, analizzandone la lunghezza d'onda e il suo comportamento come onda polarizzata.

Per determinare la misura della lunghezza d'onda abbiamo avuto modo di confrontare tre differenti metodi. Il primo, quello inerente allo studio delle onde stazionarie, ha fatto emergere la presenza di ulteriori segnali sinusoidali provocati dalla riflessione di quello principale. Questo fenomeno influiva maggiormente prendendo passi piccoli nel conteggio del numero di massimi, da qui la nostra decisione di un passo di $d = 20\,cm$. 
La misurazione della lunghezza d'onda attraverso l'esperimento di Michelson ha avuto come ostacolo principale quello di disporre di uno spazio circoscritto lungo il quale far scorrere lo specchio semi-riflettente. La variazione di cammino tenuta da noi in considerazione è stata, infatti, molto limitata; anche una singola variazione del numero dei massimi è stata causa di una grande differenza per la misurazione di $\lambda$.
L'esperimento di Lloyd è stato quello in cui abbiamo riscontrato le difficoltà maggiori. Il tentativo di determinare il primo massimo è stato, infatti, problematico. Esso era posto al di sopra del goniometro e, di conseguenza, è stato difficile determinarne la posizione precisa. A causa di ciò, anche la posizione dei successivi punti di massimo è risultata imprecisa. In aggiunta a ciò, lo spazio a disposizione era insufficiente per utilizzare correttamente l'apparato e trovare un segnale più preciso.

Per quanto concerne, invece, lo studio degli angoli di riflessione e rifrazione, i problemi riscontrati sono stati di diversa natura.
Nel caso della riflessione, abbiamo rilevato un errore maggiore in concomitanza di angoli piccoli, questo perché il ricevitore rilevava anche i segnali emessi non deviati. 
Per lo studio della rifrazione l'imprecisione più grande è stata dovuta alla determinazione dell'angolo al vertice del prisma a base triangolare. Poiché quest'ultimo presentava angoli smussati, è stato necessario riportare con precisione su un foglio la figura e, di conseguenza, determinare attraverso formule trigonometriche il valore dell'angolo.
Successivamente, nell'esperimento di polarizzazione, è stato necessario compiere due diverse interpolazioni a causa della mancata simmetria per rotazione di emettitore e ricevitore. Grazie a ciò è stato possibile verificare che il segnale rilevato dall'apparecchiatura soddisfa la legge di polarizzazione dipendendo dal $\cos^2$. 
Per la misurazione dell'angolo di Brewster, infine, abbiamo riscontrato come il valore del massimo si discosti da quello ricavato con l'interpolazione dei dati. Questo risultato è provocato dal fatto che nel fit parabolico siano stati riportati anche angoli lontani dal valore massimo. Nonostante avessimo effettuato più campionamenti vicino al massimo, è possibile che la presenza di tali angoli abbia influito sul nostro modello.



