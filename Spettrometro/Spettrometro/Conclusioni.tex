\section{Conclusioni}
Lo scopo di questo esperimento è stato quello di studiare e verificare diverse misure legate alla spettrometria attraverso l’utilizzo di due diversi oggetti analizzatori, un prisma e un reticolo dotato di diverse fenditure. Con entrambi gli strumenti abbiamo in primo luogo verificato una legge generale dell’ottica geometrica, la legge di Cauchy, nel caso del prisma, e la legge che descrive la diffrazione, nel caso del reticolo, e in secondo luogo, usando il prisma, abbiamo cercato di stimare la natura della sorgente di alcune lampade ad arco. Abbiamo condotto questa scelta perchè, come già visibile nel Paragrafo \ref{paragrafo 3}, nonostante la relazione risulti verificata, il valore caratteristico del reticolo non risulta confrontabile con quello reale e non è stato possibile individuare la natura di questi errori.

Abbiamo supposto come natura degli errori, in primis, il fatto che sia il supporto del reticolo che la componente per il fuoco del telescopio non fossero fissati rigidamente, e che, quindi, anche piccoli movimenti errati abbiano potuto provocare spostamenti di uno dei due o di entrambi, modificando le condizioni in cui sono stati raccoti i dati. Abbiamo cercato di mitigare questa inesattezza aggiungendo un termine che cercasse di correggere possibili errori nella mancata perpendicolarità tra lente e reticolo. Inoltre, sempre a causa di un malfunzionamento dello strumento, non era possibile chiudere, oltre a un certo limite, la fenditura. Ciò ha reso difficile l’osservazione dell’esperimento nel caso in cui è stato necessario tenere la lampada il più vicino possibile allo strument, com’era stato consigliato. Abbiamo, quindi, dovuto allontanare la lampada dallo spettrometro per rendere possibile le misurazione, ma perdendo così precisione nella  presa dati. Infine, le misure erano conducibili soltanto con il reticolo da $300 \frac{\text{fend.}}{\text{mm}}$, in quanto scegliendo reticoli con un numero di fenditure maggiori non era possibile acquisire più di due dati. 

Le misure condotte sulle lampade a sorgente ignota sono state più soddisfacenti e, attraverso i valori dei parametri ricavati nella verifica della legge di Cauchy, siamo arrivati a immaginare che potessero essere lampade a sodio e ad elio. D'altra parte, anche in questo caso, era presente un'inaccuratezza che supponiamo possa essere causata o, come in precedenza, dalla non voluta mobilità della componente addetta al fuoco dello strumento, oppure dalle condizioni del prisma, il quale, a causa di diversi vertici rotti o smussati, portava righe di emissione non totalmente dritte, ma leggermente incurvate. 

Un’altra fonte d’errore potrebbe essere un stima errata di $\vartheta_0$ o di $\alpha$, in questo caso però il bias dovrebbe essere presente in tutto l’esperimento, cosa che, invece, non è stata riscontrata. Nonostante tutto l’esperimento è stato soddisfacente in quanto è stato possibile verificare come le relazioni di entrambe le leggi fossero corrette e che il lavoro condotto sulle lampade ignote abbia portato un risultato coerente con lo spettro di emissione di due gas nobili.